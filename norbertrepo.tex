\documentclass[11pt]{book}              % Book class in 11 points
\parindent0pt  \parskip10pt             % make block paragraphs
\raggedright                            % do not right justify

\title{\bf THE EFFECT OF THE FERTILIZER ON OAT SEED GERMINATION IN VILLAGES}    % Supply information

\date{\today}                           %   Use current date. 

% Note that book class by default is formatted to be printed back-to-back.
\begin{document}                        % End of preamble, start of text.
\frontmatter                            % only in book class (roman page #s)
\maketitle                              % Print title page.
                  

\section{INTRODUCTION}


\subsection{BACKGROUND}
With the time rapidly approaching when we will need to grow food plants in an orbiting space
station, in a lunar base, and in a station on Mars, it would be important to know just how much fertilizer is absolutely necessary
for maximum growth in those environments. Payload size for launching support materials is very precious, so we don't want to
send up any more than necessary. Baseline data can be gathered from ground-based studies on various food plants and various
fertilizers.
Since a seed is essentially a plant in a very early stage, it is reasonable to expect that anything that helps a plant to grow should also help a seed to germinate and grow.
\subsection{ PROBLEM STATEMENT}
 Oat seeds take long to come out of the soil due to lack of manure and fertilizers. With this problem, the seeds take long to germinate as well hence leading to poverty among the village people.
\subsection{ METHEDOLOGY}
Two clear plastic vials (8.1X3.2 cm diam.) are set up, each with a strip of masking tape around it, 1 cm from
top of vial (with marks numbered 1-6, 15mm apart to mark the position of each seed, and to identify the
vial), and with a rolled-up half-piece of brown paper towel slipped inside (to hold the seeds in position
against the vial).
           The paper in each vial is moistened with a small amount of water to help hold seeds in position.
oat seeds (Avena sativa) of approximately equal length and thickness are selected. 6 will go into each vial,
as follows:
          Using forceps, each seed is grasped gently in such a way that its sharpest point is pointing away from the
hand, and just a few mm from the end of the forceps
             Working the forceps points into the space between paper towel and plastic vial, each seed is inserted to a position where its pointed (lower) end is just even with the upper edge of the tape (1 cm from lip of vial),
and just above one of the numbered marks. One vial (the "experimental") is half-filled with a tap water solution of VF-11, diluted according to directions
on bottle. The other vial (the "control") is just half-filled with tap water.
       Both vials are kept in a wood block on the window sill in room 18, at right angles to the window.
   Each day, the vials are checked for water level (maintained at 1/2 full) and seed germination.
        When green shoots begin to appear from several seeds, each will be measured (from edge of vial to upper tip of shoot) to the nearest mm, and the measurements are recorded in two data tables (one for the control vial, one for the experimental vial), designed for easy data entry and calculations.
      Both members of each research team are responsible for BOTH VIALS: measuring, recording, and watering
as needed, at each lab session.

\section{CONCLUSION}
With the help of fertilizers, and with tap water applied to the seedlings, this will help to up lift the seedlings from the soil.

                                       ATUMANYE NOBERT
                                       15/U/4010/EVE
                                        215005972


\end{document}                          % The required last line
